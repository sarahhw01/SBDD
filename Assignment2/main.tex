\documentclass[a4paper,10pt]{article}

\usepackage{chin-report}

\begin{document}
\input{header}

\begin{huge}
	\vspace{1cm}
	\textbf{Assignment Sheet 2}
\end{huge} \\

Authors: Mona Scheurenbrand, Mattes Warning, and Sarah Hüwels


\begin{large}
	\vspace{1.0cm}
	\textbf{A.2.1: Prodrugs}
\end{large}	\\ [2mm]
\begin{enumerate}
    \item What are prodrugs? \\
    A prodrug is a chemically inactive compound, that turns into an active compound in the body due to biochemical processes. In other words, a chemically active compound is modified, so that it is chemically inactive, then in vivo transformations turn the compound into the active drug again. \cite{Zawilska2013}
    The goal is to improve absorption, distribution, metabolism, excretion, and unwanted toxicity and make sure the drug works within the body at the right location at the right time. \cite{Zawilska2013}
    \item You already met medicines in the ’history’ lecture whose ’quintessence’ is a prodrug. Please give their names (they are not necessarily on the market anymore). \\
    Salicin was used as painkiller and converted in humans to salicylic acid, which was the active compound. Later it was found, that salicylic acid caused stomach problems, so it is not used as painkiller anymore.  \\
    Another prodrug from the lecture are sulfamidochrysoidin (Prontosil), which in vivo turns into sulfanilamide, which has antibacterial effects.
    \item One of these prodrugs belongs to a very versatile group of drugs and its active form can be considered a prototype of this group (hint: it contains sulfur). Which group is this? \\
    The sulfonamides belong to the group of antibacterial drugs and the active form sulfanilamide can be considered a prototype of p-aminobenzoic acid, which is essential for bacterial metabolism. 
    \item Please give the major indication(s) (lecture example) for drugs out of this group. \\
    The major indication are bacterial infections. 
    \item Briefly describe the mechanism of action of these drugs for their original indication. \\
    The original indications was using them as dyes. Here azo-dyes containing sulfonamides were used. So prontozil is an azo-dye.
    \item Why does this mechanism of action not lead to severe side-effects in humans? \\
    Because the dyes are not directly responsible for the effect. The effect is caused by the metabolite sulfanilamide. In the human body, prontosil is metabolized into sulfanilamide. 
\end{enumerate}


\begin{large}
	\vspace{1.0cm}
	\textbf{A.2.2: Receptor-Ligand Interactions}
\end{large}	\\ [2mm]


Your answer here!


\begin{large}
	\vspace{1.0cm}
	\textbf{A.2.3: Retrieving Data from the PDB}
\end{large}	\\ [2mm]


PDB_IDs: 1IJR, 2PL0, 3B2W, 3BYS, 3BYU \\
Refer to A2.3.ipynb for implementation details

\bibliographystyle{plain}
\bibliography{bib}

\end{document}
