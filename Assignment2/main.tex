\documentclass[a4paper,10pt]{article}

\usepackage{chin-report}
\usepackage{amsmath, amsthm}

\begin{document}
\input{header}

\begin{huge}
	\vspace{1cm}
	\textbf{Assignment Sheet 2}
\end{huge} \\

Authors: Mona Scheurenbrand, Mattes Warning, and Sarah Hüwels


\begin{large}
	\vspace{1.0cm}
	\textbf{A.2.1: Prodrugs}
\end{large}	\\ [2mm]
\textbf{1. [2 pt] What are prodrugs?}

Prodrugs are chemically inactive compounds, that turn into an active compound in the body due to biochemical processes. In other words, a chemically active compound is modified, so that it is chemically inactive, then in vivo transformations turn the compound into the active drug again. \cite{Zawilska2013}
The goal is to improve absorption, distribution, metabolism, excretion, and unwanted toxicity and make sure the drug works within the body at the right location at the right time. \cite{Zawilska2013}

\textbf{2. [2 pt] You already met medicines in the ’history’ lecture whose ’quintessence’ is a prodrug. Please give their names.}

Salicin was used as painkiller and converted in humans to salicylic acid, which was the active compound. Later it was found, that salicylic acid caused stomach problems, so it is not used as painkiller anymore.  

Sulfonamides such as sulfamidochrysoidin (Prontosil) are prodrugs. Prontosil is metabolized in vivo into sulfanilamide, which has antibacterial effects.

Heroin, which is a 3,6-diacetylester of morphine, is metabolized into morphine.

\textbf{3. [1 pt] One of these prodrugs belongs to a very versatile group of drugs and its active form can be considered a prototype of this group (hint: it contains sulfur). Which group is this?}

The sulfonamides belong to the group of antibacterial drugs and the active form sulfanilamide can be considered a prototype of p-aminobenzoic acid, which is essential for bacterial metabolism. 
    
\textbf{4. [1 pt] Please give the major indication(s) (lecture example) for drugs out of this group.}
    
The major indications are bacterial infections and reactive dyes. 

\textbf{5. [1 pt] Briefly describe the mechanism of action of these drugs for their original indication.}

The original indications was using them as dyes. Here azo-dyes containing sulfonamides were used. So prontozil is an azo-dye. Throughs their azo-groups, visible light can be absorbed, which gives them strong coloration. In silico, these dyes have a strong binding energy. \cite{Nayaka2024}

Mechanism in the human body: Sulfonamides are metabolized into sulfanilamide, a structural analog of p-aminobenzoic acid (PABA). By mimicking PABA, sulfanilamide competitively inhibits the enzyme dihydropteroate synthetase, which is crucial for bacterial folate synthesis and, consequently, DNA replication. \cite{sulfonamide}

\textbf{6. [1 pt] Why does this mechanism of action not lead to severe side-effects in humans?}

This mechanism does not cause severe side effects in humans because humans do not synthesize folate but obtain it from their diet. Therefore, the enzyme dihydropteroate synthetase is absent in human cells. This selective targeting of a bacterial-specific pathway allows sulfonamides to kill or inhibit bacteria without significantly affecting human cells. \cite{Dihydropteroatesynthase}

We did find some sources, that mentioned the possibility of side effects in humans. Human gut bacteria can reduce azo compounds to aromatic amines, which can be toxic for a human. So the metabolites of the azo compounds may have toxicological concerns. \cite{Josephy2023}

\begin{large}
	\vspace{1.0cm}
	\textbf{A.2.2: Receptor-Ligand Interactions}
\end{large}	\\ [2mm]


The fraction of occupied receptor binding sides is given by $\frac{[RL]}{[RL]+ [R]}$. We are interested in $\frac{[RL]}{[RL]+ [R]} = 0.5$. For this to hold we require $[RL] = [R] = [R_C]$.
\begin{proof}
We have:
\begin{align*}
\frac{[RL]}{[RL]+ [R]} &= \frac{1}{2} \\
         \frac{1}{1 +\frac{[R]}{[RL]}} &= \frac{1}{2}  \\
         1 +\frac{[R]}{[RL]} &= 2  \\
         \frac{[R]}{[RL]} &= 1  \\
         [R] &= [RL]  = [R_C]\\
\end{align*}
\end{proof}
\begin{proof}
Now we can use this in Equation 1, substituting $[RL] = [R] = [R_C]$:
\begin{align*}
K_D& = \frac{[R][L]}{[RL]} \\
K_D &= \frac{[R_C][L]}{[R_C]} \\
         K_D &= [L]
\end{align*}
$K_D$ is the (free) ligand concentration required to occupy 50\% of the receptor binding sites.
\end{proof}


\begin{large}
	\vspace{1.0cm}
	\textbf{A.2.3: Retrieving Data from the PDB}
\end{large}	\\ [2mm]


Refer to A2.3.ipynb for implementation details and the final PDB IDs.

\bibliographystyle{plain}
\bibliography{bib}

\end{document}
